\documentclass[12pt]{article}
\usepackage[utf8]{inputenc}
\usepackage[spanish]{babel}
\usepackage{siunitx}
\decimalpoint
\usepackage{mathtools}
\usepackage{amsmath}
\usepackage{amsthm}
\usepackage{amssymb}
\usepackage{graphicx}
\usepackage[margin=0.9in]{geometry}
\usepackage{fancyhdr}
\usepackage[inline]{enumitem}
\usepackage{float}
\usepackage{cancel}
\usepackage{bigints}
\usepackage{color}
\usepackage{xcolor}
\usepackage{listingsutf8}
\usepackage{algorithm}
\usepackage{tocloft}
\usepackage[none]{hyphenat}
\usepackage{graphicx}
\usepackage{grffile}
\usepackage{tabularx}
\usepackage[nottoc,notlot,notlof]{tocbibind}
\usepackage{times}
\usepackage{color}
\definecolor{gray97}{gray}{.97}
\definecolor{gray75}{gray}{.75}
\definecolor{gray45}{gray}{.45}
\renewcommand{\cftsecleader}{\cftdotfill{\cftdotsep}}
\pagestyle{fancy}
\setlength{\headheight}{15pt} 
\lhead{Examen}
\rhead{\thepage}
\lfoot{ESCOM-IPN}
\renewcommand{\footrulewidth}{0.5pt}
\setlength{\parskip}{0.5em}
\newcommand{\ve}[1]{\overrightarrow{#1}}
\newcommand{\abs}[1]{\left\lvert #1 \right\lvert}
\date{ 01 de Junio 2018}
\title{Examen}
\author{Probabilidad y estadística}

\definecolor{pblue}{rgb}{0.13,0.13,1}
\definecolor{pgreen}{rgb}{0,0.5,0}
\definecolor{pred}{rgb}{0.9,0,0}
\definecolor{pgrey}{rgb}{0.46,0.45,0.48}
\lstset{tabsize=1}

\usepackage{listings}
\lstset{ frame=Ltb,
framerule=0pt,
aboveskip=0.5cm,
framextopmargin=3pt,
framexbottommargin=3pt,
framexleftmargin=0.4cm,
framesep=0pt,
rulesep=.4pt,
backgroundcolor=\color{gray97},
rulesepcolor=\color{black},
%
stringstyle=\ttfamily,
showstringspaces = false,
basicstyle=\small\ttfamily,
commentstyle=\color{gray45},
keywordstyle=\bfseries,
%
numbers=left,
numbersep=15pt,
numberstyle=\tiny,
numberfirstline = false,
breaklines=true,
}

% minimizar fragmentado de listados
\lstnewenvironment{listing}[1][]
{\lstset{#1}\pagebreak[0]}{\pagebreak[0]}

\lstdefinestyle{consola}
{basicstyle=\scriptsize\bf\ttfamily,
backgroundcolor=\color{gray75},
}

\lstdefinestyle{Java}
{language=Java,
}

%%%%%%%%%%%%%%%%%%%%%

\lstdefinestyle{customc}{
  belowcaptionskip=1\baselineskip,
  breaklines=true,
  frame=L,
  xleftmargin=\parindent,
  language=C,
  showstringspaces=false,
  basicstyle=\footnotesize\ttfamily,
  keywordstyle=\bfseries\color{green!40!black},
  commentstyle=\itshape\color{purple!40!black},
  identifierstyle=\color{blue},
  stringstyle=\color{orange},
}

\lstdefinestyle{customasm}{
  belowcaptionskip=1\baselineskip,
  frame=L,
  xleftmargin=\parindent,
  language=[x86masm]Assembler,
  basicstyle=\footnotesize\ttfamily,
  commentstyle=\itshape\color{purple!40!black},
}

\lstset{escapechar=@,style=customc}

% Ayuda para el formato de las tablas
\usepackage{array}
% Se declara un nuevo tipo de columna para alinear de manera:
% -Horizontal
\newcolumntype{P}[1]{>{\centering\arraybackslash}p{#1}}
% -Vertical
\newcolumntype{M}[1]{>{\centering\arraybackslash}m{#1}}

% Indica la separacion entre las columnas de una tabla
\setlength{\tabcolsep}{10pt} % Default value: 6pt
% Indica el padding inferior y superior de las celdas de una tabla
\renewcommand{\arraystretch}{1.8} % Default value: 1

\usepackage{longtable}
%Permite crear columnas en el documento
\usepackage{multicol} 
\usepackage{color}
\usepackage{comment}
\newcommand{\tabitem}{~~\llap{\textbullet}~~}
\newcommand{\subtabitem}{~~~~\llap{\textbullet}~~}

\bibliographystyle{IEEEtran}
\begin{document}
		\begin{titlepage}
			\begin{center}
				
				% Upper part of the page. The '~' is needed because \\
				% only works if a paragraph has started.
				
				\noindent
				\begin{minipage}{0.5\textwidth}
					\begin{flushleft} \large
						\includegraphics[width=0.5\textwidth]{../ipn.png}
					\end{flushleft}
				\end{minipage}%
				\begin{minipage}{0.55\textwidth}
					\begin{flushright} \large
						\includegraphics[width=0.4\textwidth]{../escom.png}
					\end{flushright}
				\end{minipage}
				
				\textsc{\LARGE Instituto Politécnico Nacional}\\[0.5cm]
				
				\textsc{\Large Escuela Superior de Cómputo}\\[1cm]
				
				% Title
				
				{ \huge Sismógrafo \\[1cm] }
				
				{ \Large Unidad de aprendizaje: Instrumentación} \\[1cm]
				
				{ \Large Grupo: 3CM4 } \\[1cm]
				
				\noindent
				\begin{minipage}{0.5\textwidth}
					\begin{flushleft} \large
						\emph{Alumno(a):}\\
						
						\begin{tabular}{ll}
						Aguilar Herrera Arianna Itzamina \\
					    Diaz Ramos Enrique \\
					    Nicolás Sayago Abigail\\
					\end{tabular}
					\end{flushleft}
				\end{minipage}%
				\begin{minipage}{0.5\textwidth}
					\begin{flushright} \large
						\emph{Profesor(a):} \\
						Tellez Barrera Juan Carlos  \\
					\end{flushright}
				\end{minipage}
				
				\vfill
				
				% Bottom of the page
				{\large Fecha de inicio: 13 de Agosto de 2018} \\
				{\large Fecha de fin: Agosto de 2018}
			\end{center}
		\end{titlepage}
	
	\tableofcontents
	\newpage
	%/////////////////////////////////////////////////////////////////
	%					INTRODUCCIÓN TEORICA
	%/////////////////////////////////////////////////////////////////
	\section{Introducción teórica}
	    \subsection{Teoría sísmica}
	        \subsubsection{Uso de la instrumentación}
	         Gran parte de la observación sismológica se hace de manera instrumental. A partir de registros sísmicos instrumentales se obtienen resultados cuantitativos con base en las siguientes relaciones:
	       \begin{itemize}
	           \item Fenómenos internos como el fallamiento, movimiento del magma, explosión minera, circulación hidráulica, y fenómenos externos como el viento, la presión atmosférica, las ondas y mareas oceánicas y el ruido cultural involucran movimientos rápidos que producen ondas sísmicas detectables.
	           \item Movimientos elásticos producidos por un sistema de fuerzas.
	           \item Cuando la Tierra vibra por ondas sísmicas pasan a través de ella, a lo largo de su superficie.
	       \end{itemize}
	       A continuación se muestra las vibraciones producidas por fenómenos internos y externos y registradas instrumentalmente (Cabe destacar que la escala de tiempo para cada vibración es diferente.) 
	       \begin{figure}[H]
    	       \centering
    	        \includegraphics[scale=0.7]{Images/I1.PNG}
    	   \end{figure}
    	    
    	    \subsubsection{Instrumentos}
    	    Los instrumentos usados para observar sismos deben ser capaces de detectar la vibración pasajera, de operar continuamente con capacidad de detección muy sensitiva, poseer tiempo absoluto de tal manera que el movimiento pueda ser registrado como una función del tiempo y deben tener una respuesta lineal conocida al movimiento del suelo (instrumento calibrado) que permita que los registros sísmicos estén relacionados al contenido frecuencial y a las amplitudes del movimiento del suelo. Sin embargo, dado que no todos los instrumentos pueden registrar todos los posibles movimientos con una respuesta lineal, ha sido necesario desarrollar instrumentos para observar en el amplio rango dinámico de amplitudes y en el amplio ancho de banda en frecuencias, de todas las posibles señales de interés, evitando la interferencia de ruido ambiental.
    	    
    	    \subsubsection{Foco Sísmico}
    	    Es el lugar en tiempo y espacio donde se produce la concentración de energía y a partir del cual ésta se propaga en forma de ondas sísmicas. Con las creación del sismómetro y la instalación de las primeras redes sismológicas, empezó la determinación instrumental de los parámetros del foco sísmico. Pueden ser determinados a partir de los registros en una o varias estaciones de las ondas de cuerpo producidas por el sismo.
    	    
    	    Los parámetros que determinan el foco puntual de un sismo son:
    	    \begin{itemize}
    	        \item Coordenadas geográficas (latitud y longitud) relacionadas a un punto en la superficie (epicentro).
    	        \item Profundidad, es decir la distancia hacia el interior de la tierra a partir del epicentro. Las profundidad más el epicentro, determinan el hipocentro.
    	        \item Tiempo de origen, se refiere al momento a partir del cual se inició la liberación de energía en forma de ondas sísmicas.
    	    \end{itemize}
    	    \begin{figure}[H]
    	       \centering
    	       \includegraphics[scale=0.7]{Images/I2.PNG}
    	   \end{figure}
    	   
    	   \subsubsection{Onda sísmica}
    	   Son las perturbaciones vibracionales periódica en la que la energía se propaga a través o sobre la superficie de un medio sin translación del material Las ondas pueden ser diferenciadas por su frecuencia, amplitud, longitud de onda y velocidad de propagación. Su rango de frecuencia son de aproximadamente \textbf{1 a 100 Hz}.
	    \subsection{Filtro pasa bajas pasivo de segundo orden}
        Un filtro es un sistema que permite el paso de señales eléctricas a un  rango de frecuencias determinadas e impide el paso del resto. Son utilizados para:
        \begin{itemize}
            \item Acondicionamiento de señal de entrada.
            \item Digitalización de señales.
            \item Acondicionamiento de señal producida.
        \end{itemize}
        Un \textbf{filtro pasa bajas} es aquel que introduce muy poca atenuación a las frecuencias que son menores que una determinada frecuencia de corte. Las frecuencias que son mayores que la de corte son atenuadas fuertemente.
        
	    \subsection{Amplificadores Operacionales}
	        \subsubsection{Seguidor de Voltaje - Buffer}
	        \subsubsection{Amplificador operacional No inversor}


	%/////////////////////////////////////////////////////////////////
	%							CALCULOS
	%/////////////////////////////////////////////////////////////////
	\section{Cálculos}
		\subsection{Diseño del filtro pasa bajas activo}
		\begin{enumerate}
			\item Definición de la frecuencia de corte o $f_{c}$ o $w_{c}$, donde: $w_{c}=2 \pi f$.					\\
					Con $ f=12Hz $ tenemos que:
					$$ w_{c}=2\pi (12Hz) = 24 \pi Hz = 75.39 Hz $$
			\item Definición del capacitor 1 ($C_{1}$), eligiendo un valor comprendido entre $100pF$ y $1\mu F$.	\\
					Con $C_{1}=1\mu F$.
			\item Calcular el $C_{2}$,  $C_{2}=2C_{1}$.																\\
					Con $C_{1}=1\mu F$, tenemos que:
					$$ C_{2}=2C_{1} = 2(1\mu F) = 2 \mu F $$
			\item Calcular el resistor R, donde:																	\\
					$$ R = \frac{0.707}{w_{c}C_{1}} = \frac{0.707}{(75.39 Hz)(1 \mu F)} = \frac{0.707}{75.39\mu} = 9.37 K\Omega $$
			\newpage
		    \item Finalmente, calcular la resistencia de referencia $R_{Ref}=2R$.\\
		    Con $R=9.37 K\Omega $, tenemos que:
		    $$ R_{Ref}=2R = 2(9.37 K\Omega) = 18.75 K\Omega $$
		    

		\end{enumerate}






\end{document}