\documentclass[12pt]{article}
\usepackage[utf8]{inputenc}
\usepackage[spanish]{babel}
\usepackage{siunitx}
% ---------------------------------------------------
% 						FONT 
% ---------------------------------------------------

\usepackage{cmbright}								% Font

\decimalpoint
\usepackage{mathtools}
\usepackage{amsmath}
\usepackage{amsthm}
\usepackage{amssymb}
\usepackage{graphicx}
\usepackage[margin=0.9in]{geometry}
\usepackage{fancyhdr}
\usepackage[inline]{enumitem}
\usepackage{float}
\usepackage{cancel}
\usepackage{bigints}
\usepackage{color}
\usepackage{xcolor}
\usepackage{listingsutf8}
\usepackage{algorithm}
\usepackage{tocloft}
\usepackage[none]{hyphenat}
\usepackage{graphicx}
\usepackage{grffile}
\usepackage{tabularx}
\usepackage[nottoc,notlot,notlof]{tocbibind}
\usepackage{times}
\usepackage{color}
\definecolor{gray97}{gray}{.97}
\definecolor{gray75}{gray}{.75}
\definecolor{gray45}{gray}{.45}
\renewcommand{\cftsecleader}{\cftdotfill{\cftdotsep}}
\pagestyle{fancy}
\setlength{\headheight}{15pt} 
\lhead{Práctica 1: Sensor Sísmico (Geófono)}
\rhead{\thepage}
\lfoot{ESCOM-IPN}
\renewcommand{\footrulewidth}{0.5pt}
\setlength{\parskip}{0.5em}
\newcommand{\ve}[1]{\overrightarrow{#1}}
\newcommand{\abs}[1]{\left\lvert #1 \right\lvert}
\date{ 01 de Junio 2018}
\title{Sensor sísmico}
\author{Probabilidad y estadística}

\definecolor{pblue}{rgb}{0.13,0.13,1}
\definecolor{pgreen}{rgb}{0,0.5,0}
\definecolor{pred}{rgb}{0.9,0,0}
\definecolor{pgrey}{rgb}{0.46,0.45,0.48}
\lstset{tabsize=1}
\usepackage{wrapfig}

\usepackage{listings}
\lstset{ frame=Ltb,
framerule=0pt,
aboveskip=0.5cm,
framextopmargin=3pt,
framexbottommargin=3pt,
framexleftmargin=0.4cm,
framesep=0pt,
rulesep=.4pt,
backgroundcolor=\color{gray97},
rulesepcolor=\color{black},
%
stringstyle=\ttfamily,
showstringspaces = false,
basicstyle=\small\ttfamily,
commentstyle=\color{gray45},
keywordstyle=\bfseries,
%
numbers=left,
numbersep=15pt,
numberstyle=\tiny,
numberfirstline = false,
breaklines=true,
}

% minimizar fragmentado de listados
\lstnewenvironment{listing}[1][]
{\lstset{#1}\pagebreak[0]}{\pagebreak[0]}

\lstdefinestyle{consola}
{basicstyle=\scriptsize\bf\ttfamily,
backgroundcolor=\color{gray75},
}

\lstdefinestyle{Java}
{language=Java,
}

%%%%%%%%%%%%%%%%%%%%%

\lstdefinestyle{customc}{
  belowcaptionskip=1\baselineskip,
  breaklines=true,
  frame=L,
  xleftmargin=\parindent,
  language=C,
  showstringspaces=false,
  basicstyle=\footnotesize\ttfamily,
  keywordstyle=\bfseries\color{green!40!black},
  commentstyle=\itshape\color{purple!40!black},
  identifierstyle=\color{blue},
  stringstyle=\color{orange},
}

\lstdefinestyle{customasm}{
  belowcaptionskip=1\baselineskip,
  frame=L,
  xleftmargin=\parindent,
  language=[x86masm]Assembler,
  basicstyle=\footnotesize\ttfamily,
  commentstyle=\itshape\color{purple!40!black},
}

\lstset{escapechar=@,style=customc}

% Ayuda para el formato de las tablas
\usepackage{array}
% Se declara un nuevo tipo de columna para alinear de manera:
% -Horizontal
\newcolumntype{P}[1]{>{\centering\arraybackslash}p{#1}}
% -Vertical
\newcolumntype{M}[1]{>{\centering\arraybackslash}m{#1}}

% Indica la separacion entre las columnas de una tabla
\setlength{\tabcolsep}{10pt} % Default value: 6pt
% Indica el padding inferior y superior de las celdas de una tabla
\renewcommand{\arraystretch}{1.8} % Default value: 1

\usepackage{longtable}
%Permite crear columnas en el documento
\usepackage{multicol} 
\usepackage{color}
\usepackage{comment}
\newcommand{\tabitem}{~~\llap{\textbullet}~~}
\newcommand{\subtabitem}{~~~~\llap{\textbullet}~~}

\bibliographystyle{IEEEtran}
\begin{document}
		\begin{titlepage}
			\begin{center}
				
				% Upper part of the page. The '~' is needed because \\
				% only works if a paragraph has started.
				
				\noindent
				\begin{minipage}{0.5\textwidth}
					\begin{flushleft} \large
						\includegraphics[width=0.5\textwidth]{../ipn.png}
					\end{flushleft}
				\end{minipage}%
				\begin{minipage}{0.55\textwidth}
					\begin{flushright} \large
						\includegraphics[width=0.4\textwidth]{../escom.png}
					\end{flushright}
				\end{minipage}
				
				\textsc{\LARGE Instituto Politécnico Nacional}\\[0.5cm]
				
				\textsc{\Large Escuela Superior de Cómputo}\\[1cm]
				
				% Title
				
				{ \huge Práctica 1: Sensor Sísmico (Geófono) \\[1cm] }
				
				{ \Large Unidad de aprendizaje: Instrumentación} \\[1cm]
				
				{ \Large Grupo: 3CM4 } \\[1cm]
				
				\noindent
				\begin{minipage}{0.5\textwidth}
					\begin{flushleft} \large
						\emph{Integrantes:}\\
						
						\begin{tabular}{ll}
						Aguilar Herrera Arianna Itzamina \\
					    Nicolás Sayago Abigail\\
					    Ramos Diaz Enrique \\
					\end{tabular}
					\end{flushleft}
				\end{minipage}%
				\begin{minipage}{0.5\textwidth}
					\begin{flushright} \large
						\emph{Profesor(a):} \\
						Tellez Barrera Juan Carlos  \\
					\end{flushright}
				\end{minipage}
				
				\vfill
				
				% Bottom of the page
				{\large Fecha de entrega: 7 de Septiembre de 2018}
			\end{center}
		\end{titlepage}
	
	\tableofcontents
	\newpage
	
	% /////////////////////////////////////////////////////////////////////
	%							INTRODUCCION
	% ////////////////////////////////////////////////////////////////////
	\section{Introducción}

	% /////////////////////////////////////////////////////////////////////
	%							DESARROLLO
	% ////////////////////////////////////////////////////////////////////
	\section{Desarrollo}
		\subsection{Esquema del circuito}

		\subsection{Funcionamiento}

		\subsection{Cálculos}

		\subsection{Ajuste de valores}

	% /////////////////////////////////////////////////////////////////////
	%							MEDICIONES
	% ////////////////////////////////////////////////////////////////////
	\section{Mediciones}
		\begin{itemize}
			\item A.
			\item
			\item
		\end{itemize}
	
	% /////////////////////////////////////////////////////////////////////
	%							APLICACIÓN
	% ////////////////////////////////////////////////////////////////////
	\section{Aplicación}
		\subsection{Funcionamiento}

		\subsection{Tabla de valores}


	
	% /////////////////////////////////////////////////////////////////////
	%							CONCLUSIONES
	% ////////////////////////////////////////////////////////////////////
	\section{Conclusiones}
		\subsection{Aguilar Herrera Arianna Itzamina}

		\subsection{Nicolás Sayago Abigail}

		\subsection{Ramos Díaz Enrique}
	\end{document}
